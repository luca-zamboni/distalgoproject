
\documentclass[a4paper,12pt,notitlepage]{article} % Prepara un report per carta A4, con un font grande 12
 
\usepackage[english]{babel} % Adatta LaTeX alle convenzioni tipografiche italiane,
%\usepackage[T1]{fontenc} % Riga da togliere se si compila con PDFLaTeX
%\usepackage[utf8]{inputenc} % Consente l'uso caratteri accentati italiani
\usepackage[a4paper,top=3cm,bottom=3cm,left=2.0cm,right=2.0cm]{geometry}
\usepackage[superscript]{cite} % i numeri delle citazioni vengono messe in apice nel testo
\usepackage{babelbib} %permette di aggiungere la entry "language" nel database bibliografico
\usepackage[nottoc]{tocbibind} %aggiunge all'indice dei contenuti anche la bibliografia
\usepackage{url} % aggiunge url
\usepackage{hyperref} % aggiunge hyperref nel pdf
\usepackage{graphicx} %non mi ricordo
\usepackage{float} %per gestire la posizione delle immagini
\usepackage{amsmath} %per le formule matematiche
\usepackage{enumerate} %per liste enumerate
\usepackage[noend,boxed]{algorithm2e} % per gli algoritmi
\usepackage{tabu} %non me lo ricordo
\usepackage[format=hang, indention=-2cm,labelfont=bf]{caption} 
\usepackage{changepage} % sposta tabelle
%\usepackage{calligra}
\usepackage{fontspec} % permette di cambiare il font
\setmainfont{Calibri}
 
\frenchspacing % forza LaTeX ad una spaziatura uniforme, invece di lasciare più spazio
% alla fine dei punti fermi come da convenzione inglese

  
\begin{document}

%\pagestyle{empty}
\begin{center}
	{\bfseries\Huge {UNIVERSITY OF TRENTO}}

	\begin{center}
		\includegraphics[width=0.2\textwidth]{img/unitn}
	\end{center}
	\vspace{0.5cm}
	{\Large Project of Distributed Algorithms}
	\vspace{0.2cm}

	{ \bfseries \Large {IMPLEMENTATION OF CYCLON OVER THE AKKA FRAMEWORK}}
	\vspace{2.0cm}
	\large
	\begin{center}
		\begin{tabular}{lcl}
			%Relatore: & \hspace{5cm} &  Graduant: \\
			Luca Zamboni & \hspace{5cm} &  Luca Erculiani \\
			XXXXXX & \hspace{6cm} &  XXXXXX \\ \\
%			luca.zamboni@unitn.it & \hspace{4cm} &  luca.erculiani@unitn.it \\
		\end{tabular}
	\end{center}
	\vspace{1.0cm}
\end{center}
\begin{abstract}
   	 	
    The topic of this work is the implementation of Cyclon, a decentralized
    peer-to-peer protocol for gossiping over the Akka framework [URL AKKA REF].
    The goal of Cyclon is to build a network that can resist against crash of 
    a great part of its node without  collapsing in a series of disconnected clusters.
    This document will explain first the theoretical basis of the protocol, then 
    our implementation of it. The last part of this work will be focused on the statistical
    result of this project.
  

\end{abstract}
\newpage


% \pdfbookmark{\contentsname}{toc}
% \tableofcontents % Prepara l'indice generale
% \newpage

\section{INTRODUCTION}

The goal of this work is to implement a simulation of a peer-to-peer network running
 the Cyclon gossiping's protocol, for the project of Distributed Algorithms, by Alberto Montresor. 
 The network is strongly decentralized and consist in a 
 large number of peer clients and one or more tracker server, whose function is to allow new 
 peers to connect into the network. Now will follow a list of fast description of the topics of every
 section of this document.



\section{SYSTEM MODEL}

The system that we want to model is a dinamic collection of distribuited nodes that wants to partecipate in
 a epidemic protocol. The number of node isn't fixed and can increase or decrease depending on peers who
 join and leave the protocol, or maybe crash. 
 The communication between pairs of nodes needs that one 
 of them knows the addres of the other one, and the channel is a best-effort type (potentially a lot of 
 message omission). 


\subsection{SERVICE SPECIFICATION}

Nodes has only a partial views of the network, and this wiew is dinamic. Each node periodically 
 gossip with a random neighbour about its other neighbour. The main idea is that nodes continuously
 exchange information about other nodes, removing the old ones (so the most probable diaseppeared) and
 adding the new ones. Each node has a fixed number of neigbhbour and shuffle this with other nodes.
 For communicating with another node, a peer needs a neighbor descriptor of that node, consisting:

 \begin{itemize}
 	\item the address of that node
 	\item a timestamp information about the age of the descriptor
 	\item more additional information, maybe needed by the upper software layer[CITAZIONE MONTRE]
 \end{itemize}

When a new node want to enter in the protocol asks to a tracker 
 server a random subset of peers and start the communication.

 \subsection{SKELETON OF THE ALGORITHM}	
Here is presented the structure of the code of an instance running Cyclon. The first algorithm shown 
 is common to a bunch of protocol with the same purpose, like Newscast[CITAZIONE].

\begin{algorithm}[H]
\SetAlgoLined
\SetKwProg{Upon}{upon}{ do}{end}
\SetKwProg{UponRec}{upon receive}{ do}{end}
\SetKwProg{Repeat}{repeat}{ }{end}
\SetKw{Send}{send}
\SetKw{To}{to}

\Upon{inizialization}{
	\(view \leftarrow\) descriptor(s) of nodes already in the system
	}
\BlankLine
\Repeat{every \(\Delta\) time units}{
	Process \(q \leftarrow\) selectNeighbor(\(view\))

	\(m \leftarrow\) prepareRequest(\(view,q\))

	\Send \(\langle\mbox{REQUEST},m,p\rangle\) \To{q}
	}
\BlankLine

\UponRec{\(\langle\mbox{REQUEST},m,q\rangle\)}{
	\(m \leftarrow\) prepareReplay(\(view,q\))

	\Send \(\langle\mbox{REPLY},m',p\rangle\) \To{q}

	\(view \leftarrow\) merge(\(view,m,q\))
}
\BlankLine

\UponRec{\(\langle\mbox{REPLY},m,q\rangle\)}{
	\(view \leftarrow\) merge(\(view,m,q\))

}
\BlankLine
\end{algorithm}

What make every protocol different is the behaviour of
 the three functions called in the previous pseudocode. In Cyclon these components acts
 in this way:

\begin{itemize}
	\item selectNeighbor() selects the oldest neighobr in the view
	\item prepareRequest(\(view, q\)) removes \(t-1\) random descriptors from the view and return this 
	subset plus a fresh local one
	\item prepareReply(\(view, q\)) removes and return t freshest neighbors from the local view
	\item merge(\(view, m, q\)) merges the local view and the one received, if there are duplicates keeps only the 
	freshest. Remove itself and reinsert entries sent to \(q\) if space permits
\end{itemize}


\section{ANALYSIS AND RESULTS}

In this section we will discuss the results of simulations over the implemented protocol and some tests done under
 partarticular conditions.

\subsection{ANALYSIS METHODS}

The output of the program was used to build  graphs, using the NetworkX libraries[CITAZIONE] in Python, where
 nodes are peers and edges are the neighbors of every peer. We useed undirected graphs except for the 
 computations of in-degrees, where we obviously built a directed graph with direction peer \(\rightarrow\) neigbor.
 The switch to a graph structure allowed a series of operations, like the count of the connected components (or clusters), 
 needed to interpret the shape of the network.

We approximated a couple of opearation on the graph, in order to be able to execute it in a reasonable time. In 
 particular the computation of the clustering coefficient was made using a function of NetworkX that make an approximation[LINK TO NETX CLUSTER]. The average path length was approximated computing the average path length of a number \(n\) of random couples of nodes.


\subsection{STANDARD RUN}

In these tests we executed the simulation without interfer during the run, so no crashes or 
 communication failures were induced. This simulations were done bulding a network of 20.000 
 peers with 20 neighbors and 10 of them exchanged at every cycle. In every simulation the Cyclon
 graph is compared to a graph with the same number of nodes and the same number of edges randomly distributed.

The first simulation[FIG] compares the approximated average path length of the two graphs. We can 
 see that after few cycles the avg becames comparable with the one of the random graph. This is the 
 optimum result in a network fully decentralized without a hyerarchy.

 [IMMAGINE AVG LEN]


The second simulation 

\end{document}